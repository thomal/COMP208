The client-server architecture is neccessarily simple.

The client connects to the server, sends a single command, receives the servers
response and then disconnects. The following shows commands sent by the client,
and the servers action in response.

\begin{table}[h]
    \centering
    \begin{tabular}{lll}
    command                     & purpose             & servers action\\ \hline
    t                           & get the server time & sends back the current time (unix time in milliseconds)\\
    s   \textit{utf-8\_text}    & send messages       & the text sent is stored on the server\\
    get \textit{ms\_unix\_time} & get new messages    & every message stored since the given time is sent\\
    c   \textit{utf-8\_text}    & claim a username    & the text sent is stored on the server, with a special filename\\
    \end{tabular}
    \caption{Client-Server Protocol}
\end{table}

Every command is terminated with a linefeed. Every response from the server will
be terminated with a linefeed. The last line sent by the server will always be
"s" for success, or "e" for failure (this is ommitted from the above table).

CLAIM messages (sent with c) will be parsed by the Message class and the
username extracted for use in a filename. The filename of claim messages is as
follows
\textit{\textless unix\_time\_in\_ms\textgreater \_\textless username\textgreater};
the filename of all other messages is as follows
\textit{\textless unix\_time\_in\_ms\textgreater \_\textless SHA256\_hash\textgreater}.
