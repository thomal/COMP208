\section{Server}
\begin{lstlisting}
static void main () {
    startGUIthread()
    startServer()
}
\end{lstlisting}

\begin{lstlisting}
static void start () {
    socket = new ServerSocket(port);
    while (running) {
        incoming = socket.accept();
        t = new Thread(new Session(incoming));
        t.start();
    }
    
    shutdown();
}
\end{lstlisting}

\section{Client}
\begin{lstlisting}
static void main () {
    NetworkConnection connection    = new NetworkConnection("server.tld");
    Thread            networkThread = new Thread(connection);
    Database          db            = new Database("./db");
    GUI               gui           = new GUI(db, connection);
    Thread            guiThread     = new Thread(gui);
        
    if (!Crypto.keysExist())
        Crypto.keyGen();
        
    networkThread.start();
    guiThread.start();
        
    while (gui.isRunning())
        while (connection.hasMessage())
            Parser.parse(Crypto.decrypt(connection.getMessage()), db);
}
\end{lstlisting}

\section{Crypto}
\begin{lstlisting}
keyGen () {
    keypair = generateRSAkeypair()
    pw      = GUI.getUserInputString();
    filesystem.write("keypair", Crypto.aes(pw, keypair))
}
\end{lstlisting}

\begin{lstlisting}
static String sign (String msg) {
    byte[] sig = SHA1RSAsign(msg.getBytes("UTF-8"), Crypto.getPrivateKey());
    return Crypto.Base64Encode(sig);
}
\end{lstlisting}

\begin{lstlisting}
static String encrypt(String cmd, String text, PublicKey recipient, NetworkConnection connection) {
    Message msg = new Message(cmd, text, connection.getTime()+Crypto.rand(0,50), Crypto.sign(text));
    
    //encrypt with random AES key with random initalization vectors
    byte[]     iv = new byte[16];
    byte[] aeskey = new byte[16];
    
    fillWithRandomData(iv);
    fillWithRandomData(aeskey);
    
    byte[] aesCipherText = aes(aeskey, iv, msg.toString().getBytes("UTF-8"));
            
    //encrypt AES key with RSA
    byte[] encryptedAESKey = rsa(Crypto.getPrivateKey(), aeskey);
            
    //"iv\RSA encrypted AES key\ciper text"
    return Base64Encode(iv) + "\\" + Base64Encode(encryptedAESKey) + "\\" + Base64Encode(aesCipherText);
}
\end{lstlisting}

\begin{lstlisting}
static Message decrypt(String msg) {
    //claim messages are the only plaintext in the system, they will still need
    //decoding. Decoding here while unintuitive is easier for the rest of
    //the system because it reduces the impact of this special case.
    if (msg.substring(0,2).equals("c ")) {
        String decoding = new String(Base64Decode(msg.substring(2)));
        return Message.parse(decoding);
    }
        
    String[] tokens = new String[3];
    StringTokenizer tokenizer = new StringTokenizer(msg, "\\", false);
    tokens[0] = tokenizer.nextToken();
    tokens[1] = tokenizer.nextToken();
    tokens[2] = tokenizer.nextToken();
        
    byte[] iv            = Base64Decode(tokens[0]);
    byte[] cipheredKey   = Base64Decode(tokens[1]);
    byte[] cipherText    = Base64Decode(tokens[2]);
            
    //decrypt AES key
    Cipher rsa = Cipher.getInstance("RSA");
    rsa.init(Cipher.DECRYPT_MODE, getPrivateKey());
    byte[] aesKey = rsa.doFinal(cipheredKey);
            
    //decrypt AES Ciphertext
    SecretKeySpec aesKeySpec = new SecretKeySpec(aesKey, "AES");
    IvParameterSpec IVSpec = new IvParameterSpec(iv);
    Cipher aes = Cipher.getInstance("AES/CBC/PKCS5Padding");
    aes.init(Cipher.DECRYPT_MODE, aesKeySpec, IVSpec);
    byte[] messagePlaintext = aes.doFinal(cipherText);

    return Message.parse(messagePlaintext);
    }
\end{lstlisting}

\section{Database}

\section{Network Connection}

\section{Parser}

\section{GUI}

