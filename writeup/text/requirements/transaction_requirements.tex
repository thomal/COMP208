There are 3 categories of data transaction:
\begin{itemize}
\item 1. Data entry
\item 2. Data update and deletion
\item 3. Date queries
\end{itemize}

\subsection{Profile creation of the user}
The user is required to own an account with Turtlenet in order to log in. If
no account has been made, the user is required to create one before proceeding
to use the system. In order to create one, the user has to enter his profile
information such as his name, date of birth, gender, and other necessary
attributes. Once user has submitted it, it will be inserted into the database.

\subsection{Adding of user relations}
The term relations for this case means that the user has added has added the
other particular user into the list of the people whom he knows. Note that the
word 'friend' isn't used here mainly because a related user can fall into
different sorts of categories which is not necessarily called 'friends'.

When a user adds another user as a relation, a public key of the other user has
to be acquired first before the relationship between the two takes place. When
the public key is acquired and inserted into a field, the database will use the
key as a reference to get the details of the user. Once it is recognised, the
system then displays the user information whilst adding him into a list of
relations.

\subsection{Assigning relations into categories}
When a user adds a relation, he has a choice of adding him onto a specific
category (or categories). A user can create any category he wants by going to
the options and click 'Add new category'. The database then records the new
category into the category table.  The user then can then assign the relation
into the existing category. 

\subsection{Adding of posts}
When a user adds a new post, details such as creator of the post, content, date
and time of when the post has been created is inserted into the database. The
user has a choice of making his post either public, shared amongst a certain
category or shared with only a group of specific users. The user has the click
the names of those who can view the posts, and these names will be inserted into
the post table in the database.

\subsection{Adding of events}
This works similarly like adding a new post. However the only difference is
details such as date of the event is inserted into the database when a new event
is created. The security function is the same with post where the user can add
others to view their event. The details of these users are inserted into the
events table. 

\subsection{User creating a new message}
A user can initiate a conversation with another user by creating a new message.
When message is created, the public key is generated for user's relation to
obtain for decrypting the message. The message will then be logged by inserting
the time and date into the message table by the time the message is sent. Other
details such as the sender's and receiver's user\_id, content of the message
will be inserted into the message table as well. 

\subsection{Receiving and unlocking a message}
When the user receives a message from a relation, a notification will be sent to 
he user indicating that there is a new message. Simultaneously, this will record
the encrypted message into the message table, along with the logged details,
which is the time and date of the message. When the user enters the correct
public key, the message will the by decrypted, and this will update the message
table with the new contents of a decrypted message. 
