\section{Overview of Protocol}
Tor is an imlpementation of onion routing, it routes traffic from your computer
to an entry guard. That entry guard routes it to an internal node, which in turn
routes it to an exit node. The exit node the routes the transmission to the
final destination.\todo{how are nodes found, how are nodes public keys found,
describe hidden services and directory authorities, describe hidden services
using their public key as a URL}

RSA/AES is used to ensure that only you, the exit node, and the final
destination see the plaintext traffic being routed. With the use of TLS, SSL, or
other end-to-end encryption those who see the plaintext can be reduced to you
and the final destination. However a malicous exit node can MitM SSL connections
using ssl-strip or a similar tool. There are methods of avoiding this, but it is
a serious issue because users believe that SSL is secure.

\section{Security}
Given that Tor is a low-latency network traffic can easily be correlated. An
example of this occuring is \todo{cite lulzsec bloke who came on irc when
someone came home, and dropped off irc when they cut that someones internet
connection}

Tor does not seek to protect against size correlation of time correlation of
traffic. Rather the purview of tor is to conceal the IP address of a client from
the servers which it connects to.

Should a global passive adversary have perfect visibility of the internet, they
would be able to track tor traffic from source to host by correlating the size
and time of transmissions.

While such an attack amy initially sound infeasable, access is only needed to
the client, the host, the entry guard, the exit node, and the internal node.
So the question becomes: how likely is it that an adversary will have access to
these five nodes?

We can safely assume that the adversary has access to the clients traffic, since
our threat model is that of a nation state seeking to spy on its citizens.
Furthermore we may assume that the adversary has access to the content host, as
our threat model assumes that service operators may be pressured legally or
otherwise into spying on their users. Therefore we must instead consider how
likely is it that three random nodes will either be compromised, malicious, or
merely be in such a situation (e.g.: geographic positioning) as to have their
traffic visible to an attacker.

Given that the number of normal nodes is likely to vastly outnumber the
compromised, or malicous nodes, we can merely consider tor to be as secure as
the traffic between these three nodes.

Given that X\%\todo{stats} of tor nodes are in the US, the chance that every tor
node will lie within the US is Y\%.

Does this then mean that Tor is insecure? No. So far as we know\todo{cite tor
sucks NSA slide}the US does not currently have the ability to reliably and
consistently track tor users. This is however not something which should be
relied upon.\todo{talk about other nations, reasons to hide from US (avoid
harrassment after being 'selected')}
