The proposed project (Turtlenet) is a simple, privacy oriented social
network, which demands zero security or technical knowledge on behalf of its
users. In order to ensure security and privacy in the face of nation state
adversaries the system must be unable spy on its users even if it wants to or
server operators intend to.

We feel that obscuring the content of messages isn't enough, because suspicion
may, and often does, fall upon people not for what they say, but to whom they
are speaking\cite{trackingFriends}. Our system will therefore not merely hide
the content of messages, but the recipient of messages too. Hiding the fact that
an IP address sent a message is out of scope, but hiding which user/keypair did
so is in scope, as is which IP/user/keypair received the message and the content
of the message. It is important to hide the recipient of the message, because
otherwise they may be unfairly targeted\cite{droneCellTracking} if they use
our services to communicate with the wrong people on a phone which is later
identified, or they may merely be 'selected' for spying and harassment
\cite[3]{greenwaldAnnoying}.

We feel that current tools have significant usability problems, as was recently
made starkly clear when Glenn Greenwald, a reporter of the guardian, was unable
to work with Edward Snowden because he found GPG to be "too annoying" to use.

\begin{quote}
"It's really annoying and complicated, the [email] encryption software" - Glenn
Greenwald \cite{greenwaldAnnoying}
\end{quote}

While there exist many tools for hiding what you are saying, relatively few seek
to hide who talks with whom, and those which do often implement it merely as a
proxy, or seek to provide convenience over security.

The system is to have strict security measures implemented. It is able to
encrypt messages with the use of RSA and AES. The only way for the other user to
decrypt the data is if it was encrypted using their public key; which is given
from the recipient to the sender via whichever medium he prefers, e.g. email.
We will also allow users to transmit public keys as QR codes, for ease of use.

The system will provide a platform for people to securely communicate, both
one-to-one and in groups. Users will be able to post information to all of their
friends, or a subset of them as well as sharing links and discussing matters of
interest.

The following are our main design goals. Please note that the system is designed
with axiom that the server operators are unjust, seeking to spy on potential users, and
able to modify the source for the server.
\begin{itemize}
\item Strong cryptography protecting the content of messages
\item Make it an impossible task to derive, from the information the server has
      or is able to collect, which users send a message to which users
\item Make it an impossible task to derive, from the information the server has
      or is able to collect, which users receive a message at all
\item Transmission of public key is easy, and doesn't require knowing what a
      public key is
\item Be intuitive and easy to use, prompting the user when required
\item Provide a rich social network experience, so as to draw regular members
      and drive up network diversity
\end{itemize}

The server operator will have access to the following information:
\begin{itemize}
\item Which IP uploaded which message (although they will be ignorant of its
content, type, recipient, and sender)
\item Which IPs are connecting to the server as clients (but not what they view,
whom they talk with, or whether they receive a message at all)
\item What times a specific IP connects \footnote {While this will aid in tying
an IP address to a person, it is deemed acceptable because it is not useful
information unless the persons private key is compromised.}
\end{itemize}

A third party logging all traffic between all clients and a server will have
access to what IPs connect to the server, and whether they upload or download
information \footnote{size correlation attacks could be used here if the message
content is known}

The benefits we feel this system provides over current solutions are:
\begin{itemize}
\item Server operators can not know who talks with whom
\item Server operators can not know the content of messages
\item Server operators can not know which message is intended for which user
\item Server operators can not know who is friends with whom
\end{itemize}

In order to ensure nobody can tell who is talking with whom, we will base our
security model on the idea of shared mailboxes, as seen in practice at
alt.anonymous.messages
\footnote{https://groups.google.com/forum/\#!forum/alt.anonymous.messages}.
In this model one posts a message by encrypting it using the public key of the
recipient, and posting it in a public location. In this model, one reads a
message by downloading all messages from that location, and attempting to
decrypt them all using ones private key. Our protocol will build atop this
simple premise, and the the server will be a mere repository of messages, the
real work occurring wholly in the client.
