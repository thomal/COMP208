\section{Automated Testing}
We have included a series of automated tests with Turtlenet. These tests are
ran automatically whenever the binaries are built, they can be ran manually with
`make test'.\par

For example we can test the part of the system responsible for parsing the
contents of wall posts by constructing a wall post Message instance and calling
POSTgetKeys(). If we do not receive the expected value for any test the
developer is alerted. \par

More examples of the automated testing we used can be found at:
\url{https://github.com/thomal/COMP208/blob/master/src/ballmerpeak/turtlenet/testing/Test.java}.

\section{Black-box Testing}
Black-box testing comprised mainly of:

\begin{itemize}
    \item Clicking on buttons even if they didn't fit into the logical work flow.
    \item Attempting to 'cheat' the system by making it it do things that aren't intended by the design.
    \item Entering silly values into various parts of the system in an attempt to see if this would break anything.
\end{itemize}

The system functioned as expected in all cases. Our intent in not validating the
date field was because people don't read things telling them how to format dates
but rather type them how they normally do; however it allows for peoples date of
birth to be "the past". This perhaps isn't such a bad thing, but was ultimitly
unintentional.

\section{Usability Testing}
We had a person who had never used Turtlenet before use the system with only
five minutes of introduction beforehand. Without further instruction she was
able to easily register, add my public key that I sent her, and take part in a
conversaiton. She did however need help finding the wall of a friend. After I
showed her she found it easy to do so after.


From this we've concluded that it's preferable to have more obvious ways of
accessing anothers wall. This test occurred too late in development to add the
change.
