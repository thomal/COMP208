I think the best evaluation is that we didn't expect this. This was a thoroughly 
challenging project, that we did not anticipate being so large when we started.
Typically as a student, we found university modules to have a similar workload to 
each other through the university courses so far. This group project however was 
a different thing entirely. The work hours put in for this module eclipsed all 
other modules by far, even combined in some parts. This being said, it has also 
been rewarding to many of us, to work together producing a much bigger result than
the small projects we normally create. Self learning was also widely needed during 
this module in order to teach us all the skills required to complete the project. 
\par

Overall though, we feel the project has been a great success. With a live website 
running right now (www.turtlenet.co.uk), live working repository 
(https://github.com/thomal/COMP208), a product that works fully functioning as 
intended, its hard not for us to be satisfied. Even despite not meeting all our 
original requirements, we produced a functional social network that aptly serves
to complete our original target goal of a secure social network, that holds privacy
in its highest regard. To say things went perfectly however, would be a fallacy.
\par

There are areas we could have definitely improved upon, such as the general 
teamwork between members. The idea of group study sessions unfortunately came too
late in the project to be done, but the idea of us all taking a weekend to work on
the project together on one desk would have likely yielded positive results. 
Better work deadlines would have also been useful, more frequent, smaller workloads 
may have well served better than larger apart ones. We should have also done more 
research on what to produce the client UI with, as it turns out GWT was not the 
best choice. Apparently it is quite limiting to use and better alternatives would 
have been a more optimal choice.
\par

We did not make all bad decisions though, the idea of abandoning the provided 
university workspace in favour of git was an astounding success, and we are all
convinced reproducing the same project in that workspace would simply not be 
possible due to the complexity of the program. www.github.com provided us with the 
means to make the project possible, and documents our project from the start 
step by step, an excellent tool for group projects such as this. We are all in 
agreement version control is a must for group projects of this scale.
\par

In summary, the project was successful, and something for us as a team to be 
proud of. It is something many of us wish to continue working on even after 
this module is all over. It is something that taught us more skills than any
other module. And most of all it is something for all to have, a gift to the 
world pool of knowledge, accessible to all at no cost, and perhaps one step 
forward in a fair, free internet.
