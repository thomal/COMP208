This section extends upon the fundamentals mentioned in the Turtlenet (TN)
general section.

\section{Creating an Account}
The 'General' chapter only briefly mentions creating an account so to make this
section complete as a 'go-to' resource for users it will also be mentioned here
too.

This is where your private communications begin.

\includegraphics[scale=0.2]{screenshots/Screenshot from 2014-04-29 22-41-11}

This image shows the account creation page, which you should see when you run 
the client for the first time on your computer. From the top there are three
text boxes:
\begin{itemize}
\item a Username box
\item a Password box
\item a Confirmation box
\end{itemize}

You fill in each of the fields with the required information which will be the
following:
\begin{itemize}
\item The Username box should be filled in with your user name. This is what
      other users would call you when posting messages. This should be 
      something that represents you, but should not link to you outside of
      Turtlenet. Simply, your Turtlenet user name should not be the same as
      any other user name you use on the internet. If the name can be linked to
      you then people are able to easily determine that you have a turtlenet
      account.
\item Your password should be easy to remember but difficult for
      anyone else to guess. A good method for coming up with new passwords is
      to use four or five words, in a phrase. An example would be
      'ThisIsTurtlenetzPassword'. This is better and easier to remember than
      what is usually suggested which is a shorter password with numbers in
      them: 'P@ssw0rd'. Of course, it depends on who is remembering the
      password so choose your own method if either option mentioned feels
      uncomfortable for you.
\item The Confirmation box is where you type the password you defined in the
      previous box. Because of this, they should match, and must if the account
      creation is to be successful. The easiest way of thinking about this box
      is that it is giving you the practice of inputting your password while it
      is still fresh in your mind, to help you remember for later on.
\end{itemize}

By filling in these text boxes with the kind of information mentioned in this
section, you can then click the button underneath these boxes to create your
account. If successful you will be automatically logged in.

\pagebreak
\section{Logging into Turtlenet}
Logging into the Turtlenet client is as simple as using the password that you
had used to create your account.

\includegraphics[scale=0.2]{screenshots/Screenshot from 2014-04-29 22-24-59}

The screen shot shows the initial page you might see once you have created an
account. Enter your password into the white text box above the 'Login' button
and if the password is correct, you would have logged in.

\pagebreak
\section{Navigating around the Turtlenet client}
Getting around the client's various areas is important in order to make the most
of the functionality provided by Turtlenet. This is why all of the main
segments are provided as buttons at the top of the interface:

\includegraphics[scale=0.2]{screenshots/Screenshot from 2014-04-29 22-31-38}

The image shows that there are several main sections to the client - The wall,
the user's details, messages between the user and other people, friends that the
user has linked with and finally the function to logout. Click the
corresponding button to get to the area you wish to view. The following
sections will go through each section from right to left.

\section{Logging out}
For when you decide that you want to leave the safety of Turtlenet and work on
other things, or you simply need to be away for a while and want to be sure that
no one is using your account, you will want to log off. It is as painless as
clicking the 'log off' button found at the top right of every page. Doing so
will take you to the login screen (the one with just the password box and login
button). Of course, we wish you good fortune until you come and join us again
at Turtlenet.

\pagebreak
\section{Friends on Turtlenet}
Part of the philosophy of Turtlenet is to encrypt the messages that you send so
that only the intended recipients can read them with any understanding. These
people are known as your 'friends' on Turtlenet. In order to make any use of
Turtlenet you need to add friends. You do this by exchanging 'public keys' with
another user. Turtlenet uses Asymmetric relationships - this means that you may
have some people as friends but they might not have you as a friend. Therefore
you might understand what people have typed but they might not be reciprocated.
If this doesn't make sense at the moment, the following sections will help.

\subsection{The 'Getting'}
In order to get public keys from other users, they need to pass the information
to you.The keys can be transfered in any manner, they are not remotely private
and painting your public key on the side of your house would not diminish
security.

Once you have the public key off of your friend, you will want to proceed to the
'friends' section of the Turtlenet client, by clicking the button near the top
which has 'Friends' written upon it. You should either see the following or
something to it's effect:

\includegraphics[scale=0.2]{screenshots/Screenshot from 2014-04-29 22-31-10}

As you can see, there is 'My Key' which will be used by you to allow others to
send you messages but that will be explained in the next section. For now, you
want to click the 'Add new friend' button located to the right of the screen.
This will bring you to a screen with a long input box which asks for the key of
who will become your friend. You enter the long line of letters and numbers
that you were given by your friend into the input box. Once you have the other
person added, you should see something similar to this:

\includegraphics[scale=0.2]{screenshots/Screenshot from 2014-04-29 22-31-10}

In the image above, the current user has added themselves to their friends list.
Simply repeat the process as it takes you back to the main section for the
friends tab.

\subsection{The 'Making'}
By getting other people's keys you can send messages to them but for people to
send anything back that you can read, they would need to have your key as well.
All you do in order to help others add you is to send the letters and numbers
in the text box next to 'My key' and get the other user to follow the steps in
the above section 'The 'Getting'.'

\subsection{Banding Together}
In Turtlenet you can associate other users with categories, custom made by you.
This is useful if you want to send the same message to a number of people.
To do this, whilst you are in the friends section of the client, click the 
'Create category' button on the right. It should take you to this screen:

\includegraphics[scale=0.2]{screenshots/new6}

You will give your category a name so it hints to the kind of users you have in
them together by typing the group name in the top text box. Click 'Create
category' once you have finished the naming procedure. You are then able to add
any members you wish whose keys you have attached to your account. This is done
in the drop-down menu at the bottom of the interface and then clicking the 'Add
friend' button next to said menu. If you no longer want a particular user in
the group any more, select their user name in the large box in the middle and
click the button to the side which says 'Remove from group'.

\pagebreak
\section{Messages in Turtlenet}
Messages can be sent to singular users or they can be sent to categories of
users created by the current user of the client. Below is an example of what
you may find in the messages section:

\includegraphics[scale=0.2]{screenshots/Screenshot from 2014-04-29 22-30-50}

\begin{itemize}
\item the box at the left hand side is for your available recipients - our
      example user only has himself at the moment. This will fill up over time
      when you add public keys from other users.
\item The larger of the two boxes is where you type the content of your message.
      There is no size limit.
\item The Send button on the right finalises the message and sends it to the
      recipient to read. You cannot edit your message once you have sent it
      so be sure to re-read what has been typed to avoid any unfortunate errors!
\end{itemize}

\section{What's mine is mine - Personal Data}
When using Turtlenet, personal data is just that - personal. Similar to all of
the messages and posts you make, your personal data is also encrypted and made
secure so that the server moderators have no access to them. Here is a view at
what you could see when entering the 'My Details' section of the client:

\includegraphics[scale=0.2]{screenshots/Screenshot from 2014-04-29 22-43-08}

The image shows the only personal information that you may store using the
Turtlenet client. Note that the only piece of information here that is 
important is the user name - all other fields are optional and at the user's
discretion to fill in or not. Each button to the right saves what is currently
in the associated field at the time of clicking, so you will need to save again
if you edit after a save.

Below these fields is a list of categories you have created, check the box next
to a category and members of it will be able to see your personal data.
Unchecking the box hides any futures changes in it from them.

A note about revoking your key:  This means that you mark your key as never
again to be trusted, and so messages from it are ignored.
\textbf{Do not click unless you wish to erase your Turtlenet presence.}
After a revocation, another key is made for you to use, which means that any other
users that had your key will need to be informed that you have changed and you
will need to give them your new key if you wish to continue getting messages 
and posts from them.

\pagebreak
\section{Personal Graffiti - your Turtlenet wall}
Your wall is a central social hub for many users of Turtlenet. It is a
collection of messages aimed at the user, who may be off-line at the time.
This section is for the functionality of the wall.

In Turtlenet a post is the generic way of talking about a message being left for
another user - think of it similar to a sticky note on a cork board.An example
of a wall is below:

\includegraphics[scale=0.2]{screenshots/Screenshot from 2014-04-29 22-26-44}

The image outlines a couple of posts being made by the example user. Before 
posting is explained, this manual will explain the other elements in view:

\begin{itemize}
\item The 'About me' button allows a user to see an overview of their personal
      data. This allows a user quick access to their key, which could be sent to
      another user.
\item You can 'like' posts to show enjoyment, appreciation or agreement with
      what another user has posted. This is done by simply clicked the 'like'
      that is found underneath the target post. Should your political views
      change for example, you can unlike any currently liked post in the same
      manner - clicking the 'unlike' that will be found in the same place under
      the target post.
\item Commenting on a post is also possible with the Turtlenet client. Simply
      click the 'Add a comment' phrase underneath the target post and a large
      input box will appear beneath. Simply type your 'two cents' then click
      the 'Post comment' button under the input box. If you decide not to
      insert an interjection then you may click the Cancel button to remove the
      box and not attach your comment to the post.
\end{itemize}

Posting is as simple as clicking the 'Write a post' button near the top, which
will bring a couple of new elements into the client:

\includegraphics[scale=0.2]{screenshots/Screenshot from 2014-04-29 22-46-29}

As the above image shows, there are a couple of new buttons and a large text box
that appears onto the Turtlenet client. First it is easiest to define a target
for the post, which is done by clicking the drop down menu below the input box
on the left side. The user is able to choose from categories
that have been created, sending the post to multiple users. Once decided, type
the content of the post into the large input box. Once finished, click the
'Send' button below the input box on the right side. If you wish to stop
making a post, click the Cancel button in the middle, underneath the input box.
