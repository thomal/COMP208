\section{Frequently Asked Questions}
This is the section which should hopefully answer most of the questions that
most users might have about the system.  Sending emails to one of the addresses 
in the contact section in the beginning of the user manual may help you get your
answer but it is best if you continue looking for an answer whilst you wait for
an official reply.

\subsection{What does Turtlenet do?}
Think of Turtlenet in a similar manner to any other social network commonly in
use.  It allows users to communicate with each other and allowing other people
to voice their opinions on what others have written.  At the moment it is text
based, meaning you can't attach images and video to it when you post or comment.
You can however send links to such content to each other.  That is a convenient
enough work around for the time being as it
means that no one is having to download an encrypted video but are never able to
view it as they do not have the key to unlock the data.  I think everyone will
appreciate not having to download hundreds of copies of current top 40 each week.

\subsection{How many accounts can I have on Turtlenet?}
You should only need one, but we don't preclude you from having more. If you
merely wish to seperate the content people can see then categories are a better
solution, and future versions of Turtlenet will allowing the sharing of
different personal information with different groups. If you simply must have
multiple accounts though, there's nothing stopping you. Just launch the client
in a different directory.

\subsection{I forgot my password.  Can someone reset it for me?}
The short answer is no.  Turtlenet was designed so that no one but the user had
any access to their account.  As
a result, if you lose your password we are unable to recover anything in the
account.  The only thing you can do is simply to create another.  Feel safe in 
the knowledge that everything is encrypted on your old account so at least no 
one can access what was lost except those people you already shared it with.

\subsection{Where is everything stored?}
Information is stored on your computer, laptop or whatever else it is that uses the Turtlenet client.
Each client downloads all of the data and reads what it can, using keys you have
collected over time off of other users.  Keeping it local means that no readable is
stored on the server, so evil moderators cannot have their way with your data.
Encrypted data is stored on the server, but nobody can read it who you didn't
send it to.

\subsection{How big does this database get?}
As the only things being stored are text, not images or video, this means that
each message is only small and will likely be less than a few megabytes over one
year's very active use.

\subsection{Why would someone want to build from source?}
Given that compatability of jars isn't an issue the only reason to do so is to
ensure that your binaries derive from the public source code and not an evil
secret version.

\subsection{The Client does stuff I don't think it should do...}
You may have found a bug for us accidentally.  email to one of the addresses at
the beginning of the user manual and the developers will have a look at it.  As
the source is being released, maybe the community will have a look and suggest a
fix themselves.

\subsection{What do Server Moderators of Turtlenet do?}
We don't have any, we can't moderate content we can't see.

\subsection{I want to mod Turtlenet.  Can I have the source?}
It's nice to know that others wish to take up the helm, pioneering a secure 
method of communication.  You can have the source, it is available to the public
to browse and modify.

\subsection{Why choose 'X' over the clearly superior 'Y'?}
As developers ourselves, we understand that other people have differing opinions.
That's the joy of releasing code.  Other people can pick up what we have done,
or use our ideals as a starting point for their own thing.  What this project
stood for is ease of use for the end user and security from any unwanted external
influences and this, we believe, is achieved.
