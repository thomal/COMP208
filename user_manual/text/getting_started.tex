\section{Getting started}
Welcome to using Turtlenet!  Through the use of Turtlenet, you will experience
the ease of use and the practicality of communicating and socialising with your
friends, family, business associates or anyone else that you know through a
medium where your data is ensured to be protected.  This user manual has been
designed and written specifically to assist the users by providing detailed
description of all the various uses of the program.  Let's get started!

\section{System Requirements}
These are the minimum system requirements for Turtlenet:

\begin{itemize}
\item An internet connection
\item Any OS with a JRE (version 1.6.x or higher)
\item Any up-to-date browser
\end{itemize}

\section{Installing Turtlenet}
In order to install Turtlenet, you simply download ONE of the files from our
website:

www.turtlenet.co.uk/downloads.html

Most users will want to get the version that is without 'TOR' as unless you know
what that acronym stands for, you won't have it installed.  It is an external
piece of networking software which adds another layer of security, hiding your
IP address so people don't know where you currently are.

As the file is a Java Archive (JAR), you can put it in whatever folder you
choose - Turtlenet doesn't mind.  It will create the required files and folders
when it is running so just pick a pleasant home for the download.

\section{Running Turtlenet}
Now you have the client on your computer, you will need to run it.  People who
are familiar in using Java may be able to work it out but this section is here
for those who want to make sure that they are going to run it first time
correctly and without frustration.  Here is what you do:
\begin{enumerate}
\item Open Command Prompt (Windows) or your Terminal (*nix and OS X)
\item use 'cd' to get to where your Turtlenet client .jar file is.
      Windows users changing drive letters will need the '/D' parameter.
      e.g. 'cd /D D:\\TurtlenetFolder\\'
\item You will want to run the java command: \newline
      \textit{'java -jar turtlenet.jar'}
\end{enumerate}

If you managed to get to the downloaded client JAR file and ran that command,
you should have the back end of the Turtlenet client running.  All you need to
do now is open your preferred browser, or one of the suggested browsers if you
have more than one, and type \textit{'localhost:3141'} into your URL bar.

If the browser did not complain about anything and just worked, you should see
a Turtlenet banner.  If so, you have your client running successfully!

\section{The Turtlenet Interface}
Turtlenet comes with a simple interface that has the main menu, which has the
following sections:
\begin{itemize}
\item My Wall
\item My Details
\item Messages
\item Friends
\item Logout
\end{itemize}

\section{Account Creation}
The user is expected to create a new account when using Turtlenet for the first
time.  In order to create an account, enter a user name and a password, as well
as repeating your password into the confirmation box.  Once the user has 
created an account they will be logged into Turtlenet.  From here onwards, the
user can then add further profile details should they wish to. How to do so will
be explained under the 'Using the System' section.
